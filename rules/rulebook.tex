%% LyX 2.4.4 created this file.  For more info, see https://www.lyx.org/.
%% Do not edit unless you really know what you are doing.
\documentclass[american,twocolumn]{article}
\usepackage[T1]{fontenc}
\usepackage[utf8]{inputenc}

\makeatletter

%%%%%%%%%%%%%%%%%%%%%%%%%%%%%% LyX specific LaTeX commands.
\newcommand{\noun}[1]{\textsc{#1}}

%%%%%%%%%%%%%%%%%%%%%%%%%%%%%% User specified LaTeX commands.
% Enumerate numbering 
\usepackage{enumitem}

\setlist[enumerate,1]{%
  label=\thesubsubsection.\arabic*,
  ref=\thesubsubsection.\arabic*,
  leftmargin=*,
}

\makeatletter
\@addtoreset{enumi}{subsubsection}
\makeatother

% Footer 
\usepackage{fancyhdr}
\pagestyle{fancy}
\usepackage{datetime2}
\DTMsetdatestyle{iso}

\fancyhf{} % clear all header/footer fields

\fancyfoot[C]{Rulebook v0.2.1 — \DTMtoday}
\fancyfoot[R]{\thepage}

\renewcommand{\headrulewidth}{0pt}
\renewcommand{\footrulewidth}{0.4pt}

\makeatother

\usepackage{babel}
\begin{document}

\section{Overview}

\subsection{Description}
\begin{enumerate}
\item This is an economic, network-building game. Players control shares
in companies and act for those companies , expanding their network
to increase their share value and to funnel money back into the players'
own pockets. Shares are auctioned throughout the game. Shifting cross-investments
result in shared and conflicting incentives across the table. The
game end is triggered when either a) all shares of one company have
been auctioned off or b) all players pass consecutively.
\end{enumerate}

\subsection{Key Concepts}

\subsubsection{Map}\label{subsec:Map}
\begin{enumerate}
\item The map is made up of abstracted \noun{Hexes}.There are four types
of hexes: starting (black), mountain (gray), standard (white), and
city (yellow).
\end{enumerate}

\subsubsection{Network}\label{subsec:Network}
\begin{enumerate}
\item A company-owned hex is indicated by a stack of 1-5 chips with the
controlling company’s cube on top. A company’s \noun{Network} is the
collection of all hexes they own.
\item A given hex’s value\label{enu:Hex-value} is equal to the number of
chips on the hex.
\item All companies are considered to own the starting (central, black)
hex, which is considered to have a hex value of 5.
\item Besides the starting hex, only one company may own any given hex.
I.e., a standard or city hex may only be part of one company’s network.
\item The term \noun{Network Value\label{enu:Network-Value}} is used to
describe the combined value of a collection of hexes.
\item A company is considered to have access to a hex if the hex is adjacent
to any hex in the company’s network.
\item A company may lease network (\ref{enu:Leasing-Network}) from one
or more other companies to gain access to a hex it otherwise does
not have access to.
\end{enumerate}

\subsubsection{Shares}\label{subsec:Shares}
\begin{enumerate}
\item A company's \noun{Share Value\label{enu:Share-value}} is equal to
the total network value (\ref{enu:Network-Value}) of all hexes owned
by the company. E.g., at the start of the game, all companies have
a share value of \$5.
\item \label{enu:Share-auctioned-status}Each Share is dual sided. One side
has an empty circle on it, indicating it is unauctioned. The opposite
side has a red circle with a bar through it, indicating it has been
auctioned already. A share that has been auctioned may not be auctioned
again.
\item \label{enu:Share-control-own}A player \noun{controls} all shares
in their player area, but only \noun{own} shares on their auctioned
side.
\end{enumerate}

\section{Setup}\label{sec:Setup}

\subsection{Map Setup}
\begin{enumerate}
\item Place the map in the center of the table.
\end{enumerate}

\subsection{Company Setup}
\begin{enumerate}
\item For each player in the game, prepare one company
\begin{enumerate}
\item Put one share above the map to act as the company charter.
\item Place \$5 on the company charter as the company’s starting treasury.
\end{enumerate}
\item Return remaining companies to the box. They will not be used this
game.
\end{enumerate}

\subsection{Player Setup}\label{subsec:Player-Setup}
\begin{enumerate}
\item Each player takes all 5 remaining shares of one company in the game
and spreads them out in front of them, 4 on their unauctioned side
and 1 on its auctioned side (\ref{enu:Share-auctioned-status}).
\item Each player takes \$25 in starting cash.
\end{enumerate}

\section{Gameplay}\label{sec:Gameplay}

\subsection{Turn Structure}
\begin{enumerate}
\item On a turn, a player may take one of the following actions:
\begin{enumerate}
\item Auction (\ref{subsec:Auction})
\item Act for a Company (\ref{subsec:Act-for-company})
\item Pass (\ref{subsec:Pass})
\end{enumerate}
\item After a player has taken one of these, the player to their left takes
a turn.
\item Play continues in this clockwise fashion until the Game End is triggered
(\ref{subsec:Game-End-Triggers}).
\end{enumerate}

\subsection{Auction}\label{subsec:Auction}
\begin{enumerate}
\item When auctioning, the turn player takes either one unauctioned share
they control (\ref{enu:Share-control-own}) or any share in the bank
pool and starts an auction.
\item The turn player is considered the current bidder and may optionally
make an initial bid.
\begin{enumerate}
\item The minimum bid must be at least the given company’s share value (\ref{enu:Share-value})
and no greater than the turn player’s cash.
\item The turn player does not have to place a bid to start an auction.
\end{enumerate}
\item The player to the left of the current bidder now may either place
a bid or pass.
\begin{enumerate}
\item If no bid has been placed yet, the new bid must be at least the given
company’s share value (\ref{enu:Share-value}).
\item If a bid has already been placed, the new bid must be greater than
the current bid.
\item Players cannot bid more than they have in cash.
\item If a player passes, they cannot join back in on subsequent rounds.
\end{enumerate}
\item Play continues in this clockwise fashion, until either all players
have passed or all players besides one bidder have passed.
\begin{enumerate}
\item If all players pass (no bids are placed), put the auction share in
the bank, and . Do not flip the share over to its auctioned side as
it may still be auctioned in the future.
\item If only one bidder remains, the winning bidder (the player who last
bid, the only player who did not pass) wins the auction.
\begin{enumerate}
\item The winning player takes the auctioned share, flips it over to indicate
it has been auctioned, and places it in front of them with the other
shares they control.
\item If the share was auctioned from the bank pool, place the winning bid
into the bank. Otherwise, place the winning bid into the auctioned
share’s company’s treasury.
\end{enumerate}
\end{enumerate}
\item If the auctioned share was the last auctioned share for a given company,
the game end is triggered (\ref{subsec:Game-End-Triggers}).
\end{enumerate}

\subsection{Act for a Company}\label{subsec:Act-for-company}
\begin{enumerate}
\item When acting for a company, a player \noun{Expands Network} for that
company, potentially triggering \noun{payouts} for that company and/or
any company they leased network from, and finally perform \noun{upkeep}.
\item In order to act for a company, a player must control (\ref{enu:Share-control-own})
at least one share in said company. A player may only act for a single
company on a given turn. When acting for a company, all money spent
must come from the given company’s treasury. If the company cannot
afford an action, it cannot take said action.
\item To expand network onto a given hex, the player choose a desired hex
value (\ref{enu:Hex-value}) for the target hex and pays \$5 times
that value to the bank from the acting company’s treasury. Place chips
on the hex to indicate the hex value (e.g., 2 chips for a 2-value
hex), and place a cube of the acting company on top of the chip stack
to indicate ownership. In order to expand to hexes a company does
not immediately have access to, it may lease network from a combination
of other companies (\ref{enu:Leasing-Network}). If the company has
not reached its expansion limit (\ref{enu:Expansion-limit}), it may
continue to expand.
\item If the company has expanded onto a city hex, all owners of shares
of the acting company receive an immediate payout per share equal
to the given cities hex value (\ref{enu:Hex-value}). E.g., if a company
expands onto a city hex, chooses a hex value of 2, and Player A has
4 shares in the given company, they would receive a payout of \$8
(4 x \$2).
\item Some restrictions apply:
\begin{enumerate}
\item A company may only expand onto unowned hexes. That is, a company may
not expand onto a hex already part of its own network (can’t increase
the value of an already owned hex), nor can it expand onto a hex owned
by another company (not to be confused with leasing network from another
company).
\item \label{enu:Expansion-limit}The network value of the newly expanded
network (i.e., the value of the network built this turn, not the total
company network) may not be greater than the number of shares the
given company has auctioned so far this game. E.g., at the start of
the game, prior to any auctions, any given company may only expand
to a single 1-value hex.
\item Mountain hexes can never be part of a network (cannot be expanded
onto).
\item A company may only expand onto one hex type (standard or city) on
a given turn.
\end{enumerate}
\item Leasing Network\label{enu:Leasing-Network}
\begin{enumerate}
\item A company may lease network from other companies in order to expand
onto a hex it otherwise does not have access to by tracing a route
from any part of its own network through the leased network(s) to
an empty hex.
\item The route through the leased network(s) must follow the shortest route
from any combination of companies to the target hex. Therefore, if
a company already has access to a given hex, it may not lease network
to said hex.
\item The cost of leasing network is the combined network value of the hexes
used. The company must pay this value to the bank. If they cannot,
they cannot take the action. If after leasing network, the company
does not have enough money to expand onto the target hex, they cannot
take that action.
\item After leasing network, all owners of shares for companies of the leased
network(s) receive a payout per share of the network value of their
company’s leased network. E.g., if the acting company leased network
of a total value of 2 from the green company and a total value of
3 from the yellow company, and Player A had 2 shares in the green
company and 1 in the yellow company, they would receive a payout of
\$7 (Green: 2 x \$2, Yellow: 1 x \$3, Total: \$7).
\end{enumerate}
\item Upkeep
\begin{enumerate}
\item For each unauctioned share they control, they must pay \$1 to the
share company’s treasury.
\item If they cannot afford to pay this upkeep cost, they must put one of
their unauctioned shares into the bank pool, and take the company's
current share value (\ref{enu:Share-value}) from the bank and put
in the company's treasury
\end{enumerate}
\end{enumerate}

\subsection{Pass}\label{subsec:Pass}
\begin{enumerate}
\item If a player chooses to pass, they do nothing, and play continues to
the next player in turn order. If all players pass consecutively,
the game ends immediately (\ref{subsec:Game-End-Triggers}).
\end{enumerate}

\section{Game End}\label{sec:Game-End}

\subsection{Triggers}\label{subsec:Game-End-Triggers}
\begin{enumerate}
\item The game ends when one of the following conditions have been met:
\begin{enumerate}
\item A company has had all of its shares auctioned off
\begin{enumerate}
\item In this event, all players besides the current turn players (the player
who triggered the game's end) get one more turn, and then a final
round is played, starting with the player who triggered the end game.
\end{enumerate}
\item All players pass consecutively
\begin{enumerate}
\item In this event, the game ends immediately.
\end{enumerate}
\end{enumerate}
\end{enumerate}

\subsection{Scoring}
\begin{enumerate}
\item The winner of the game is the player with the highest \noun{Net Worth}.
A players net worth is the sum of the following:
\begin{enumerate}
\item Their cash
\item The Share Value of all their owned shares (\ref{enu:Share-control-own}).
\begin{enumerate}
\item Unauctioned shares are worth nothing.
\end{enumerate}
\end{enumerate}
\end{enumerate}

\end{document}
