%% LyX 2.4.4 created this file.  For more info, see https://www.lyx.org/.
%% Do not edit unless you really know what you are doing.
\documentclass[american,twocolumn]{article}
\usepackage[T1]{fontenc}
\usepackage[utf8]{inputenc}

\makeatletter

%%%%%%%%%%%%%%%%%%%%%%%%%%%%%% LyX specific LaTeX commands.
\newcommand{\noun}[1]{\textsc{#1}}

%%%%%%%%%%%%%%%%%%%%%%%%%%%%%% User specified LaTeX commands.
% Enumerate numbering 
\usepackage{enumitem}

\setlist[enumerate,1]{%
  label=\thesubsubsection.\arabic*,
  ref=\thesubsubsection.\arabic*,
  leftmargin=*,
}

\makeatletter
\@addtoreset{enumi}{subsubsection}
\makeatother

% Footer 
\usepackage{fancyhdr}
\pagestyle{fancy}
\usepackage{datetime2}
\DTMsetdatestyle{iso}

\fancyhf{} % clear all header/footer fields

\fancyfoot[C]{Rulebook v0.2.2 — \DTMtoday}
\fancyfoot[R]{\thepage}

\renewcommand{\headrulewidth}{0pt}
\renewcommand{\footrulewidth}{0.4pt}

\makeatother

\usepackage{babel}
\begin{document}

\section{Overview}

\subsection{Description}
\begin{enumerate}
\item This is an economic, network-building game. Players are investors
in competing wireless infrastructure companies. To fund operations,
players issue shares, enabling companies to expand their network,
secure access to new areas by building towers, and generate payouts
by expanding to cities. As networks grow in value, so do the value
of the companies' shares. However, this comes with risk as issuing
shares dilutes initial equity and creates openings for opportunistic
cross-investors seeking to funnel that money into their own pockets
via leasing rival companies' infrastructures.\\
\\
The game end is triggered when either a) all shares of one company
have been issued or b) all players pass consecutively.
\end{enumerate}

\subsection{Key Concepts}

\subsubsection{Map}\label{subsec:Map}
\begin{enumerate}
\item The map is made up of abstracted \noun{Hexes}, representing areas
of coverage. There are four types of hexes: starting (black), mountain
(gray), standard (white), and city (yellow).
\end{enumerate}

\subsubsection{Network}\label{subsec:Network}
\begin{enumerate}
\item A company-controlled hex represents exclusive rights to infrastructure
in that area. Control is indicated by a stack of 1-5 chips with the
company’s cube on top. A company’s \noun{Network} is the collection
of all hexes it controls.
\item A given hex’s value\label{enu:Hex-value} is equal to the number of
chips on the hex.
\item All companies are considered to control the starting (central, black)
hex, which has a hex value of \$5.
\item Besides the starting hex, only one company may control any given hex.
I.e., a standard or city hex may only be part of one company’s network.
\item The term \noun{Network Value\label{enu:Network-Value}} is used to
describe the combined value of a collection of hexes.
\item \label{enu:hex-access}A company has access to a hex if the hex is
adjacent to any hex in its network.
\item A company may lease network (\ref{subsec:Leasing-Network}) from one
or more other companies to gain access to a hex it otherwise does
not have access to.
\end{enumerate}

\subsubsection{Shares}\label{subsec:Shares}
\begin{enumerate}
\item A company's \noun{Share Value\label{enu:Share-value}} is equal to
the total network value (\ref{enu:Network-Value}) of all hexes it
controls. E.g., at the start of the game, all companies have a share
value of \$5.
\item \label{enu:Share-auctioned-status}Each share is dual sided. The side
that is currently face up indicates what type of share it is:
\begin{enumerate}
\item One side has an empty circle on it, indicating initial investor equity.
These represent unissued shares that have not been sold to raise company
capital.
\item The opposite side has a red circle with a bar through it, indicating
an issued share. Once a share has been issued, it may not be issued
again.
\end{enumerate}
\item \label{enu:Share-control-own}A player \noun{controls} all shares
in their player area, but only issued shares count towards final scoring.
\end{enumerate}

\subsubsection{Payouts}\label{subsec:Payouts}
\begin{enumerate}
\item Players make money when shares they control receive payouts. This
happens in two scenarios:
\begin{enumerate}
\item When a company expands onto a city hex (\ref{enu:city-payout})
\item When a company's network is leased during a rival company's expansion
(\ref{enu:leased-network-payout})
\end{enumerate}
\end{enumerate}

\section{Setup}\label{sec:Setup}

\subsection{Map Setup}
\begin{enumerate}
\item Place the map in the center of the table.
\end{enumerate}

\subsection{Company Setup}
\begin{enumerate}
\item For each player in the game, prepare one company
\begin{enumerate}
\item Put one share above the map to act as the company charter.
\item Place \$5 on the company charter as the company’s starting treasury.
\end{enumerate}
\item Return remaining companies to the box. They will not be used this
game.
\end{enumerate}

\subsection{Player Setup}\label{subsec:Player-Setup}
\begin{enumerate}
\item Each player takes all 5 remaining shares of one company in the game
and places them in their player area: 4 on their unissued side, and
1 on its issued side (\ref{enu:Share-auctioned-status}).
\item Each player takes \$25 in starting cash.
\end{enumerate}

\section{Gameplay}\label{sec:Gameplay}

\subsection{Turn Structure}
\begin{enumerate}
\item On a turn, a player may take one of the following actions:
\begin{enumerate}
\item Auction (\ref{subsec:Auction})
\item Act for a Company (\ref{subsec:Act-for-company})
\item Pass (\ref{subsec:Pass})
\end{enumerate}
\item After a player has taken one of these, the player to their left takes
a turn.
\item Play continues in this clockwise fashion until the game end is triggered
(\ref{subsec:Game-End-Triggers}).
\end{enumerate}

\subsection{Auction}\label{subsec:Auction}
\begin{enumerate}
\item When auctioning, the turn player selects either one unissued share
they control (\ref{enu:Share-control-own}) or any share in the bank
pool and starts an auction.
\item The turn player is considered the current bidder and may optionally
make an initial bid, and then in clockwise order, players may bid
or pass
\begin{enumerate}
\item The initial bid must be at least the given company’s current share
value (\ref{enu:Share-value}).
\item If a bid has already been placed, the new bid must be greater than
the current bid.
\item A player may not bid more than their available cash.
\item If a player passes, they cannot join back in on subsequent rounds.
\end{enumerate}
\item Play continues in this clockwise fashion, until either all players
have passed or only one bidder remains.
\begin{enumerate}
\item If all players pass (no bids are placed), put the auctioned share
in the bank pool, and place money from the bank into the company's
treasury equal to its current share value.
\item If only one bidder remains, that player wins the auction.
\begin{enumerate}
\item The winning player takes the auctioned share, flips it to its issued
side, and places it in their player area with their other controlled
shares.
\item If the share was auctioned from the bank pool, place the winning bid
into the bank. Otherwise, place the winning bid into the auctioned
share’s company’s treasury.
\end{enumerate}
\end{enumerate}
\item If this was the last unissued share for a given company, the game
end is triggered (\ref{subsec:Game-End-Triggers}).
\end{enumerate}

\subsection{Act for a Company}\label{subsec:Act-for-company}
\begin{enumerate}
\item To act for a company, a player must control at least one share of
that company, issued or otherwise(\ref{enu:Share-control-own}).
\item When acting for a company, players expand the company's network by
building towers to control hexes.
\begin{enumerate}
\item A player may only act for one company on a given turn.
\end{enumerate}
\item All money spent for expanding company network comes from the company's
treasury.
\begin{enumerate}
\item If the company cannot afford an action, it cannot take that action.
\end{enumerate}
\item The total value of hexes expanded to on a given turn may not exceed
the number of shares issued for the company.
\item When expanding network:
\begin{enumerate}
\item Choose a target hex the company has access to (\ref{enu:hex-access})
that no company controls.
\item Choose a hex value (\ref{enu:Hex-value}).
\item Pay \$5 x that value from the company treasury to the bank.
\item Stack a number of chips equal to the chosen value on the target hex.
\item Place one cube for the company on top of the stack of chips, indicating
the company now controls that hex.
\item \label{enu:city-payout}If the company expands into a city hex, all
players who control shares in the acting company receive from the
bank an immediate payout per share equal to the target city hex's
value.
\end{enumerate}
\item Finally perform Upkeep
\begin{enumerate}
\item For each unissued share the turn player controls, they must pay \$1
to the share company’s treasury. This represents the cost of retaining
ownership that has not yet been sold to raise capital.
\begin{enumerate}
\item This comes from the player's personal treasury, not the company they're
acting for.
\item They must pay this cost regardless of which company they acted for
this turn. E.g., if they acted for a company they only own issued
shares in, they still must pay upkeep on their unissued shares.
\end{enumerate}
\item If they cannot afford to pay this cost, they must:
\begin{enumerate}
\item Put one of their unissued shares into the bank pool
\item Take the company's current share value (\ref{enu:Share-value}) from
the bank and put it in the company's treasury.
\end{enumerate}
\end{enumerate}
\end{enumerate}

\subsubsection{Leasing Network}\label{subsec:Leasing-Network}
\begin{enumerate}
\item A company may lease network from other companies in order to expand
onto a hex it otherwise does not have access to (\ref{enu:hex-access})
by tracing a route from any part of its own network through the leased
network(s) to an empty hex.
\item The route through the leased network(s) must follow the shortest route
from any combination of networks to the target hex. Therefore, if
a company already has access to a given hex, it may not lease network
to said hex.
\item The cost of leasing network is the combined network value of the hexes
used. The company must pay this value to the bank.
\begin{enumerate}
\item If they cannot afford this cost, they cannot take the action.
\item If after leasing network, the company does not have enough money to
expand onto the target hex, they cannot take the action.
\end{enumerate}
\item \label{enu:leased-network-payout}After leasing network, all players
who control a share for a company whose network was leased receive
a payout per share of the total value of the company's leased network.
\end{enumerate}

\subsection{Pass}\label{subsec:Pass}
\begin{enumerate}
\item If a player chooses to pass, they do nothing, and play continues to
the next player in turn order.
\item If all players pass consecutively, the game ends immediately (\ref{subsec:Game-End-Triggers}).
\end{enumerate}

\section{Game End}\label{sec:Game-End}

\subsection{Triggers}\label{subsec:Game-End-Triggers}
\begin{enumerate}
\item The game ends when one of the following conditions have been met:
\begin{enumerate}
\item All shares of a single company have been issued
\begin{enumerate}
\item In this event, all players besides the current turn players (the player
who triggered the game's end) get one more turn, and then a final
round is played, starting with the player who triggered the end game.
\end{enumerate}
\item All players pass consecutively
\begin{enumerate}
\item In this event, the game ends immediately.
\end{enumerate}
\end{enumerate}
\end{enumerate}

\subsection{Scoring}
\begin{enumerate}
\item The winner of the game is the player with the highest \noun{Net Worth}.
A players net worth is the sum of the following:
\begin{enumerate}
\item Their cash
\item The Share Value of all issued shares they control (\ref{enu:Share-control-own}).
\begin{enumerate}
\item Unissued shares are worth nothing.
\end{enumerate}
\end{enumerate}
\end{enumerate}

\end{document}
